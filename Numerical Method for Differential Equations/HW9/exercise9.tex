\documentclass{ctexart}

\usepackage{amsmath}

\usepackage{amsthm}

\usepackage{amssymb}

\usepackage{bm}

\usepackage{graphicx}

\usepackage{listings}
\lstset{
basicstyle=\scriptsize
}

\usepackage{caption}

\begin{titlepage}

\title{微分方程数值解 \\ 第九周作业}

\author{于慧倩 \\ 14300180118}

\date{2017年4月}

\end{titlepage}

\begin{document}

\maketitle

\newpage

\begin{enumerate}
%第一题
\item P151.2

\begin{enumerate}
\item 证明(3.1.54) 如果有\(e_N=e_0=0\),则\( \| \bm{e}\|_{\ell_2} \leq \|\delta_x^+ \bm{e}\|_{\ell_2} \)

\begin{proof}

\begin{eqnarray*}
e_i &=& e_i-e_{i-1}+e_{i-1} - e_{i-1} +\dots +e_1-e_0\\
&=&\sum_{j=0}^{i-1} (e_{j+1}-e_j)\\
&=&\sum_{j=0}^{i-1} h \delta_x^+ e_j
\end{eqnarray*}

\[\mbox{令} u'(x) = \left \{ \begin{array}{rl}
 \delta_x^+ e_{[ \frac{x}{h}]} &\mbox{ if $\frac{x}{h} \in N^*$} \\
  0 &\mbox{ otherwise}
       \end{array} \right.
\]
\begin{eqnarray*}
\|\bm{e}\|_{\ell_2}^2 &=& \sum_{i=1}^{N-1} e_i^2\\
&=&\sum_{i=1}^{N-1} (\sum_{j=0}^{i-1} h \delta_x^+ e_j)^2\\
&=& \sum_{i=1}^{N-1}( \int_0^{(i-1)h} u' (s) \mbox{d} s)^2\\
&=& \frac{1}{h^2} \int_0^1( \int_0^x u' (s) \mbox{d} s)^2 \mbox{d}x\\
&\leq& \frac{1}{h^2} \int_0^1(x \int_0^x u'(s)^2\mbox{d}s)\mbox{d}x\\
&\leq&\frac{1}{2h^2} \int_0^1u'(x)\mbox{d}x
\end{eqnarray*}

\begin{eqnarray*}
\| \delta_x^+ \bm{e}\|_{\ell_2}^2 &=& \sum_{i=1}^{N-1} (\delta_x^+ e_i)^2\\
&=& \frac{1}{h^2}\int_0^1u' (x)^2 \mbox{d} x
\end{eqnarray*}

故有\(\bm{e}\|_{\ell_2}^2 \leq \delta_x^+ \bm{e}\|_{\ell_2}^2 \)

\end{proof}

\item  证明收敛性和两阶收敛
有截断误差
\[
R_i=-\delta_x^2 u(x_i)-f(x_i)\]

定义函数\(e_i=u(x_i)-u_i\),则有

\[R_i=-\delta_x^2 e_i\]

两边同时乘\(e_i\),得到
\[
-e_i \delta_x^2 e_i=R_ie_i\]

则有

\[
\| \delta_x^+ e\|^2_{l^2}=\sum_{i=1}^{N-1} R_ie_i \leq \|R\|_{l^2}\|e\|_{l^2}
\]

由于上已经证明\(\|e\|_{l^2} \leq \| \delta_x^+ e \|_{l^2}\)

所以得到

\[
\|e\|_{l^2} \leq \|R\|_{l^2}
\]

同时有截断误差
\[
R_i = -\frac{h^2}{12}u^{(4)}(\theta_i) 
\]

所以有\(\|e\|_{l^2}\)收敛,得证三点差分格式的收敛性。

得\(e_i\)二阶收敛。

导数值\(\delta_x^+ u_i=\delta_x^+[u(x_i)-e_i]\),其收敛性即\(\delta_x^+ e_i\)的收敛性。

有
\[
 \| \delta_x^+ e \| ^2_{l^2} \leq \|R\|_{l^2} \| e \|_{l^2} \leq \|R\|_{l^2} 
\]
所以有导数也二阶收敛。

\end{enumerate}

\item P151.3

\begin{enumerate}
\item 证明三点差分格式的极值原理仍然成立

有相应的三点差分格式:
\[
-\frac{a(x_n)u_{n+1}-(a(x_n)+a(x_{n-1}))u_n+a(x_{n-1})u_{n-1}}{h^2}=f_n
\]

展开即
\[
a(x_n)u_{n+1}=(a(x_n)+a(x_{n-1}))u_n-a(x_{n-1})u_{n-1}-h^2f_n
\]

不妨假设极值原理不成立,\(u_n\)为三点差分格式求出的最小值
\[
u_{n+1} \geq u_n , u_n \geq u_{n-1}
\]

同时有
\begin{eqnarray}
a(x_n)(u_{n+1}-u_n)=a(x_{n-1})(u_n-u_{n-1})-h^2f_n
\end{eqnarray}
由于
\[
f_n \geq 0,a(x)\geq \alpha >0
\]
所以有,\((1)\)的左边\(\geq 0\),右边\(\leq 0\),所以若\(u_n\)达到最小值,应有\(u\)恒为常数。所以有极值原理成立。


\item 给出三点差分格式的最大模误差估计

三点差分格式离散得到的截断误差为:
\[
R_n = - \delta_x^-[a(x_n)\delta_x^+u(x_n)]-f(x_n)\]

定义函数\(e_i=u(x_i)-u_i\),则有\(-\delta_x^-[a(x_n)\delta_x^+e_n)]=R_n=-\frac{h^2}{12}u^{(4)}(x_n)+O(h^4)\)
 
 而\(a(x) \geq \alpha \geq 0\)所以\( \| - \delta^2_x e_n\|_{l^\infty} \leq \|R_n\|_{l^\infty}/ \alpha\),构造\(v_0=v_N=0\)且\(-\delta^2_x v_n=\|R\|_{l^\infty}/\alpha\)
,则有\(v_n=\frac{n}{2N}(1-\frac{n}{N})\|R\|_{l^\infty}/\alpha\).最终得到\(u(x)\)的四阶导数一致有界为\(M_4\),则有\(\|e\|_{l^\infty}=\frac{M_4h^2}{96\alpha}\)

\end{enumerate}
\end{enumerate}
\end{document}