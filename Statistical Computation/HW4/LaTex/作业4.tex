\documentclass{ctexart}

\usepackage{amsmath}

\usepackage{amsthm}

\usepackage{amssymb}

\usepackage{mathabx}

\usepackage{bm}

\usepackage{graphicx}

\usepackage{listings}
\lstset{
basicstyle=\scriptsize
}

\usepackage{caption}

\begin{titlepage}

\title{统计中的计算方法 \\ 第四次作业}

\author{于慧倩 \\ 14300180118}

\date{2017年6月}

\end{titlepage}

\begin{document}

\maketitle

\newpage

\section{第一题}

假设随机变量X和Y都在\((0,B)\)上取值。假设\(f(x|y)=C(y) e^{-xy},0<x<B  f(y|x)=C(x)e^{-xy},0<y<B\),给出一种方法来近似模拟X,Y,并用模拟的方法来
估计\(E(X),E(XY)\)

\subsection{模拟方法}

首先通过\(\int_0^B f(x|y)=C(y) e^{-xy}=1\)计算出,\(C(y)=\frac{y}{1-e^{-By}}\).

同理:\(C(x)=\frac{x}{1-e^{-Bx}}\).

选择用Gibbs采样进行模拟:
\begin{itemize} 
\item Step 1:随机给出初始值,\((x_0,y_0) \in (0,B)\times(0,B)\)
\item Step 2:重复取样\(x_i  \sim f(x|y_{i-1})\),\(y_i \sim f(y|x_i)\);
\end{itemize}

其中生成\(x_i  \sim f(x|y_{i-1})\),采用inverse方法进行:
\begin{itemize} 
\item Step 1:生成\((0,1)\)均匀分布的随机数U
\item Step 2:\(x_i=-\frac{1}{y} \ln (1-\frac{Uy}{C(y)})\);
\end{itemize}

生成\(y_i\)的步骤同理。

\subsection{模拟估计}

调用MCMC.Gibbsxy.R中的函数,生成\(10^4 \)个随机变量且B时,
MCMC.Gibbsxy(\(10^4\),1):
得到\(E(X)\)如下表:
\begin{table}[h]
\centering
\begin{tabular}{| c |c| }
          \hline
          \bf E(X) &\bf E(XY) \\
          \hline
  0.4596391 &0.2047234 \\
          \hline
 0.4610378 & 0.2064006\\
          \hline
       0.4615507 & 0.2033755\\
          \hline
 0.4651505 &0.2075964\\
\hline
0.468295 & 0.208894\\
\hline
\end{tabular}
\end{table}

\section{第二题}

假设 \(Xi, i = 1, 2, 3\),相互独立且服从均值为 1 的指数分布。设计一种模拟方法来估计,\(E(X1+2X2+3X3 |X1+2X2+3X3 >15),E(X1+2X2+3X3 |X1+2X2+3X3 <1)\)

分别用Gibbs采样与Metropolis-Hastings 算法生成两种分布:

\subsection{\(E(X1+2X2+3X3|X1+2X2+3X3 >15)\)}

运用Gibbs采样方法:
\begin{itemize} 
\item Step 1:选取初值\((X_1,X_2,X_3)\),使得\(X1+2X2+3X3 >15)\)
\item Step 2:重复操作,从三个变量中选出两个,\(I,J\),固定另一个不变,对\(I、J\)进行变化;
\end{itemize}

其中假设固定\(X_3\)不变,意识到有\(x_i\)相互独立且服从均值为1的指数分布且\(X_1+2X_2>15-3X_3\)。
计算条件概率,
\[ f(X_1|X_1+2X_2>15-3X_3=a) = C e^{-1\times X_1} \int^{\infty}_{(a-X_1)/2} e^{-1\times X_2} \mbox{d} X_2 =C' e^{-1 \times X_1 +1/2 X_1} \]

生成参数为\(1-1/2\)的指数分布的随机数,令\(X_1\)等于这个数,然后再用inverse方法计算\(X_2\)。

\[f(X_2|X_2>15-3X_3-X_1=b)=e^{b-X}I_{X\geq b}\]
\[F(X_2|X_2>15-3X_3-X_1=b)=(1-e^{b-X})I_{X\geq b}=U\]

运用inverse方法生成\(X_2\):
\begin{itemize} 
\item Step 1:生成(0,1)上均匀分布的随机数\(U\)
\item Step 2:令\(X_2=b-\ln (1-U)=b-\ln (U)\)
\end{itemize}

\subsection{模拟估计}

见\(H4\_2\_1.R\)

选取初值为(5,5,5),调用\(H4\_2\_1(10^5)\),生成\(10^5\)个随机数,去除前1000个,运行得到\(E(X_1+2X_2+3X_3 | X_1+2X_2+3X_3 >15)\)如下表:
\begin{table}[h]
\centering
\begin{tabular}{| c | }
          \hline
          \bf $E(X_1+2X_2+3X_3 | X_1+2X_2+3X_3 >15)$  \\
          \hline
   17.66761 \\
          \hline
 17.65996\\
          \hline
      17.67592\\
          \hline
 17.68168\\
\hline
17.66473\\
\hline
\end{tabular}
\end{table}
\subsection{\(E(X_1+2X_2+3X_3 |X_1+2X_2+3X_3 <1)\)}

用Metropolis-Hastings 算法生成,Proposal Kernel采用Random Walk:

\[
\theta'=\theta+Z,Z\sim N(0,\sigma^2)
\]

概率分布
\[f(X_1+2X_2+3X_3 |X_1+2X_2+3X_3 <1)=Ce^{-(X_1+X_2+X_3)}\times I_{(X_1+2X2+3X3 <1)}\times I_{X_1>0}\times I_{X_2>0}\times I_{X_2>0}\]

单次循环中接受概率:
\[\alpha(X^{(i-1),X*}) = exp(\sum(X^{(i-1)})-\sum(X*)) \times I_{(X*_1+2X2+3X3 <1)}\times I_{X_1>0}\times I_{X_2>0}\times I_{X_2>0}\]

具体算法如下:
\begin{itemize} 
\item Step 1:选取适当的初值,保证\(X_1+2X_2+3X_3 <1\)
\item Step 2:生成\(Z \sim N(0,\sigma^2)\),令\(X^*=X+Z\)
\item Step 3:计算单次接受概率\(\alpha(X^{(i-1),X^*})\)
\item Step 4:生成(0,1)上均匀分布的随机变量U,若\(U<\alpha(X^{(i-1),X^*})\),令\(X^{(i)}=X^*\),否则保持\(X^{(i)}=X^{(i-1)}\)
\end{itemize}

\subsection{模拟估计}

见\(H4\_2\_2.R\)

选取初始值为\((0.1,0.1,0.1)\),为保证接受率选取\(\sigma=0.09\),调用\(H4\_2\_2(10^4)\),生成\(10^4\)个随机数,去除前1000个,得到\(E(X_1+2X_2+3X_3 |X_1+2X_2+3X_3 <1)\)如下表
\begin{table}[h]
\centering
\begin{tabular}{| c | }
          \hline
          \bf $E(X_1+2X_2+3X_3 | X_1+2X_2+3X_3 <1)$  \\
          \hline
0.7340738 \\
          \hline
 0.7261952\\
          \hline
   0.727004\\
          \hline
0.7210271\\
\hline
0.7357822\\
\hline
\end{tabular}
\end{table}


\section{第三题}

假设X,Y,Z的联合概率密度为\(f(x,y,x)=C e^{-(x+y+z+axy+bxz+cyz)},x>0,y>0,z>0\),其中\(a,b,c\)为非负常数,C取值与\(a,b,c\)无关,估计X,Y,Z。并估计E(XYZ),当\(a=b=c=1\)时。

\subsection{模拟方法}

采用Metropolis-Hastings算法估计X,Y,Z:

Proposal Kernel采用Random Walk:

\[
\theta'=\theta+Z,Z\sim N(0,\sigma^2)
\]

概率分布
\[f(x,y,x)=C e^{-(x+y+z+axy+bxz+cyz)},x>0,y>0,z>0\]

单次循环的接受概率:
\begin{eqnarray*}
& &\alpha(X^{(i-1)},X*)\\
&=&exp((X+Y+Z+aXY+bXZ+cYZ)-(X^*+Y^*+Z^*+aX^*Y^*+bX^*Z^*+cY^*Z^*))\\
& &\times I_{X^*>0}\times I_{Y^*>0}\times I_{Z^*>0}
\end{eqnarray*}

具体算法如下:
\begin{itemize} 
\item Step 1:选取适当的初值,保证\(X_1>0,X_2>0,X_3>0\)
\item Step 2:生成\(Z \sim N(0,\sigma^2)\),令\(X^*=X+Z\)
\item Step 3:计算单次接受概率\(\alpha(X^{(i-1),X^*})\)
\item Step 4:生成(0,1)上均匀分布的随机变量U,若\(U<\alpha(X^{(i-1),X^*})\),令\(X^{(i)}=X^*\),否则保持\(X^{(i)}=X^{(i-1)}\)
\end{itemize}

\subsection{模拟估计}

见MCMC.Metroxyz.R

调用MCMC.Metroxyz.R(1,1,1),选取初始值为\((1,1,1)\),为保证接受率选取\(\sigma=0.3\),生成\(10^4\)个随机数,去除前1000个,得到\(E(X_1X_2X_3)\)如下表所示
   
   
\begin{table}[h]
\centering
\begin{tabular}{| c | }
          \hline
          \bf $E(X_1 X_2 X_3)$ \\
          \hline
   0.0840624  \\
          \hline
 0.08302989 \\
          \hline
       0.09016883\\
          \hline
 0.09280316\\
\hline
0.1034397\\
\hline
\end{tabular}
\end{table}

   

\section{第四题}

\subsection{模拟方法}

假设X,Y,N的联合分布为
\begin{eqnarray*}
&& P(X=i,y \leq Y \leq y+dy,N=n) \\
&\approx& C{n \choose i}y^{i+\alpha-1}(1-y)^{n-i+\beta-1}e^{-\lambda}\frac{\lambda^n}{n!}dy
\end{eqnarray*}
其中,i= 0, ... ,n;n = 0,1, ... ;y $≥ $0.当$\alpha$=2,$\beta$=3,$\lambda$=4时,用模拟方法估计$ E(X),E(Y),E(N)$

首先有
$$F(y+dy)-F(y)=\int_{y}^{y+dy}f(x,t,n)dt=C{n \choose i}y^{i+1}(1-y)^{n-i+2}e^{-\lambda}\frac{\lambda^n}{n!}dy$$
所以有
$$\frac{F(y+dy)-F(y)}{dy}=C{n \choose i}y^{i+1}(1-y)^{n-i+2}e^{-\lambda}\frac{\lambda^n}{n!}$$

令$dy\rightarrow0$得到

$$P(X=i,Y=y,N=n)C{n \choose i}y^{i+1}(1-y)^{n-i+2}e^{-\lambda}\frac{\lambda^n}{n!}$$

再求$X,Y,N$分别的条件概率

$$P(X=i|Y=y,N=n)=\frac{ {n \choose i}y^i (1-y)^{-i}}{\sum_{i=0}^{i=n} {n \choose i}y^i (1-y)^{-i}}  $$

$$P(Y=y|X=i,N=n)=\frac{y^{i+1}(1-y)^{n-i+2}}{\Gamma(i+2) \Gamma(n-i+2)/\Gamma(n+5)}$$

$\lambda=4(1-y)$:
$$P(N=n|X=i,Y=y)=\frac{e^{-4(1-y)} [4(1-y)]^{n-i}}{(n-i)!}
$$
可以看出P(X=i|Y=y,N=n)是二项分布,P(Y=y|X=i,N=n)是Beta分布,P(N=n|X=i,Y=y)看作是$(n-i)$的泊松分布,

选择用Gibbs采样进行模拟:
\begin{itemize} 
\item Step 1:随机给出初始值,\((x_0,y_0,n_0) \)并且满足条件
\item Step 2:选出随机数$\in$(1,2,3),对应进行采样。取样方式为:\(x_i  \sim f(x|y_{i-1},n_{i-1})\),\(y_i \sim f(y|x_i,n_{i-1})\),$n_i \sim f(n|x_i,y_i)$,并且保证$x=0,\dots,n ;0\leq y\leq 1;n=0,\dots,$;
\end{itemize}

\subsection{模拟估计}

调用函数\(H4\_4(10000,2,3,4)\)五次得到的均值如下:

\begin{table}[h]
\centering
\begin{tabular}{| l | c | r | }
          \hline
          \bf E(X) & \bf E(Y) & \bf E(N) \\
          \hline
        1.5511 &0.3952789 &3.9191\\
          \hline
  1.5537 &0.3960138 &3.9497\\
          \hline
        1.6074 &0.4058824 &3.9492\\
          \hline
1.5895& 0.3958827 &4.0418\\
\hline
1.5077& 0.385433 & 3.9543\\
\hline
\end{tabular}
\end{table}

\section{第五题}

生成两个二维正态分布生成的混合正态分布,两个二维正态分布的均值和协方差矩阵为\((1,4),(-2,-1);\),两个分布中随机变量产生的概率分别为0.5,0.5。

\subsection{模拟方法}
采用Metropolis-Hastings算法得到两个单独的二维正态分布链,然后再进行随机混合,使得两个分布中随机变量产生的概率分别为0.5,0.5。

概率分布
\[f(x_j,y_j)=\frac{1}{2\pi \sqrt(\Sigma_j)}exp(-\frac{1}{2}(x-\mu_j)^T\Sigma_j^{-1}(x-\mu_j))\]

单次循环的接受概率:
\[\alpha_j(X^{(i-1)},X*)=exp(\frac{1}{2}(x-\mu_j)^T\Sigma_j^{-1}(x-\mu_j)-\frac{1}{2}(x*-\mu_j)^T\Sigma_j^{-1}(x*-\mu_j))
\]

具体算法如下:
\begin{itemize} 
\item Step 1:选取适当的初值,保证\(X_j>0,Y_j>0\)
\item Step 2:生成\(Z \sim N(0,\sigma_j^2)\),令\(X_j^*=X_j+Z\)
\item Step 3:计算单次接受概率\(\alpha_j(X^{(i-1),X^*})\)
\item Step 4:生成(0,1)上均匀分布的随机变量U,若\(U<\alpha_j(X^{(i-1),X^*})\),令\(X_j^{(i)}=X_j^*\),否则保持\(X_j^{(i)}=X_j^{(i-1)}\)
\end{itemize}

如此分别得到两个二维正态分布的随机数,在随机生成\((0,1)\)上均匀分布的随机数U,若\(U<1/2\),令\(X=X_1\),否则\(X=X_2\)。这样得到两个二维正态分布的混合分布。

\subsection{模拟估计}

见\(H4\_5.R\)

这里选取初值为二维正态分布的均值,共计生成5000个随机数,去掉前1000个,得到数据如图:

\centerline{\includegraphics[width=5.5in]{H4_5.eps}}

\end{document}
