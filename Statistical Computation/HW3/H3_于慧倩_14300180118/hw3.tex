\documentclass{ctexart}

\usepackage{amsmath}

\usepackage{amsthm}

\usepackage{amssymb}

\usepackage{mathabx}

\usepackage{bm}

\usepackage{graphicx}

\usepackage{listings}
\lstset{
basicstyle=\scriptsize
}

\usepackage{caption}

\begin{titlepage}

\title{统计中的计算方法 \\ 第三次作业}

\author{于慧倩 \\ 14300180118}

\date{2017年5月}

\end{titlepage}

\begin{document}

\maketitle

\newpage



%第一题
\section{第一题}

用两种方法产生以下分布,并进行数值试验:
\[ P{X=i} = \frac{e^{-\lambda}\lambda^i/i!}{\sum_{j=0}^{k} e^{-\lambda} \lambda^j/j!},  i=0,\dots,k \]

\subsection{method 1: the inverse transform method}
 \subsubsection{计算方法}
生成在\([0,1]\)均匀分布的随机数\(U\),继而生成:

\[
X = \left\{ \begin{array}{rl}
  x_0 &\mbox{ if \(U<p_0\)} \\
  x_1 &\mbox{ if \(p_0 \leq U < p_0+p_1\)} \\
  \dots &\dots \\
  x_k  &\mbox{ if \(\sum_{i=0}^{k-1}p_i \leq U < \sum_{i=0}^k p_i\)}
       \end{array} \right.
\]

通过下述方法生成\(X\)

\begin{itemize} 
\item 生成随机数\(U\)
\item 如果\(U < p_0\),令\(X=x_0\),停止。
\item 否则,如果\(U < p_0+p_1\),令\(X=x_1\),停止。
\item \dots
\item 否则,如果\(U < \sum_{i=0}^{k-1}\),令\(X=x_{k-1}\),停止。
\item 否则,令\(X=x_k\),停止。
\end{itemize}

由于\(p_i=P\{x=i\}=e^{-\lambda} \frac{\lambda^i}{i!}/ \sum_{j=0}^k e^{-\lambda} \frac{\lambda^j}{j!}\),我们有\(p_{i+1}=p_i \frac{\lambda}{i+1}, i\geq 0\),所以实际上操作如下

\begin{itemize} 
\item Step 1:计算\(C=\sum_{j=0}^k e^{-\lambda} \frac{\lambda^j}{j!}\)
\item Step 2: 生成随机数\(U\)
\item Step 3: \(i=0,p=e^{-\lambda}/C,F=p\)
\item Step 4: 如果 \(U \leq F\),令\(X=i\),停止
\item Step 5: \(i=i+1;p=\lambda p/i;F=F+p\);
\item Step 6:重新回到Step 4继续判断
\end{itemize}


\subsubsection{数值试验}

取\(k=10,\lambda = 5\)为例进行数值试验,得到结果如下图所示。重复操作20000次得到长度为20000的分布如题干所示的随机数\(X\),且均值、方差分别为\(4.8859,4.435303\)

\centerline{\includegraphics[width=5.5in]{H3_1.eps}}



\subsection{method 2:the acceptance-rejection method}
 \subsubsection{计算方法}
取Y为平均分布,即其概率密度函数\(\{q_j=\frac{1}{k}\},j \geq 0\),令\(c = max(\frac{p_j}{q_j})\),计算可得\(c = \frac{k}{C}[\lambda-1]\)。

通过下述方法生成\(X\)
\begin{itemize} 
\item Step 1:生成随机数\(U_1\),令\(Y=[kU_1]\)
\item Step 2:生成随机数\(U_2\)
\item Step 3:如果\(U_2 \leq \frac{p_Y*k}{c}\),令\(X=Y\),停止。否则回到Step 1.
\end{itemize}


\subsubsection{数值试验}

仍然取\(k=10,\lambda = 5\)为例进行数值试验,得到结果如下图所示。重复操作10万次得到长度为24462的分布如题干所示的随机数\(X\),且均值、方差分别为\(4.816205,4.03032\)。并且\(\frac{1}{c}=0.02465762\),实际有效操作比例为\(0.024462\)

\centerline{\includegraphics[width=5.5in]{H3_2.eps}}



\section{第二题}

给出模拟下述分布的方法,并进行数值试验:
\[
p_j = \left\{ \begin{array}{rl}
  0.11 &\mbox{ j is odd , and \(5 \leq j \leq 13\)} \\
  0.09 &\mbox{ j is even, and \(6 \leq j \leq 14\)} \\
     \end{array} \right.
\]
\subsection{method 1:the inverse transform method}
 \subsubsection{计算方法}
通过下述方法生成\(X\)

\begin{itemize} 
\item Step 1:生成随机数\(U\)
\item Step 2: \(p=0.11\),令\(x=x_0=5\),如果\(U < p\),令\(X=x\),停止。

\item Step 3:否则,\(p = p+0.09\),\(x=x+1\),如果\(U < p\),令\(X=x\),停止。
\item Step 4:否则,\(p = p+0.11\),\(x=x+1\),如果\(U < p \),令\(X=x\),停止。
\item Step 5:否则,回到Step 3。
\end{itemize}


\subsubsection{数值试验}

进行数值试验,得到结果如下图所示。重复操作2万次得到长度为2万的分布如题干所示的随机数\(X\),且均值、方差分别为\(9.45085,8.242596\)。

\centerline{\includegraphics[width=5.5in]{H3_2_1.eps}}


\subsection{method 2:the acceptance-rejection method}
 \subsubsection{计算方法}
取Y为平均分布,即其概率密度函数\(\{q_j=\frac{1}{10}\},j \geq 0\),令\(c = max(\frac{p_j}{q_j})=1.1\)。

通过下述方法生成\(X\)
\begin{itemize} 
\item Step 1:生成随机数\(U_1\),令\(Y=[10U_1]+5\)
\item Step 2:生成随机数\(U_2\)
\item Step 3:如果\(U_2 \leq \frac{p_Y}{0.11}\),令\(X=Y\),停止。否则回到Step 1.
\end{itemize}

\subsubsection{数值试验}

进行数值试验,得到结果如下图所示。重复操作10万次得到长度为91020的分布如题干所示的随机数\(X\),且均值、方差分别为\(9.451362,8.221225\)。并且\(\frac{1}{c}=0.9090909\),实际有效操作比例为\(0.9102\)

\centerline{\includegraphics[width=5.5in]{H3_2_2.eps}}




\section{第三题}  

给出具有如下概率密度函数的随机变量的产生方法,并进行数值验证:
 \[
f(x)= \left\{ \begin{array}{rl}
  e^{2x} & -\infty < x< 0\\
  e^{-2x} & 0<x<\infty \\
     \end{array} \right.
\]

\subsection{the inverse transform algorithm}
 \subsubsection{计算方法}
因为\(U\)为\((0,1)\)上均匀分布,当\(x<0\)时,有\(\int_{-\infty}^x e^{2t} \mbox{d} t =U\),得到\(x=\frac{1}{2} \ln (2U)\)。当\(x
\geq 0\)时,有\(\int_{-\infty}^0 e^{2t} \mbox{d} t +\int_0^x e^{2t} \mbox{d}t=U\),得到\(x=-\frac{1}{2} \ln (2(1-U))\)


,按照下述方法生成\(X\):

\begin{itemize} 
\item Step 1:生成\((0,1)\)上均匀分布随机数\(U_1\) 
\item Step 3:生成\((0,1)\)上均匀分布随机数\(U_2\) 
\item Step 4: if \(U_2 \leq \frac{1}{2}\),令\(X=\frac{1}{2} \ln (2U_1)\),否则\(X=-\frac{1}{2} \ln (2(1-U_1))\)
\end{itemize}

\subsubsection{数值试验}

进行数值试验,得到结果如下图所示。

a)重复操作10万次

得到长度为10万的分布如题干所示的随机数\(X\),且均值、方差分别为\(0.001559111,0.2755747\)。

\centerline{\includegraphics[width=5.5in]{H3_3.eps}}

b)重复操作1万次

得到长度为1万的分布如题干所示的随机数\(X\),且均值、方差分别为\(-0.005648216,0.2654377\)。

\centerline{\includegraphics[width=5.5in]{H3_3_2.eps}}

c)重复操作1千次

得到长度为1千的分布如题干所示的随机数\(X\),且均值、方差分别为\(0.006721561,0.2410665\)。

\centerline{\includegraphics[width=5.5in]{H3_3_3.eps}}





\section{第四题}  

给出具有如下概率密度函数的随机变量的产生方法,进行数值试验, 并讨论方法的运算效率:\(f(x) =30(x^2−2x^3+x^4),0 \leq x\leq1\).

\subsection{method 1: the acceptance-rejection method}


\subsubsection{计算方法}

令\(Y\)为均匀分布,其概率密度函数\(g(x)=1,  0<x<1\),这样有\(c= max (\frac{f(x)}{g(x)})=\frac{15}{8}\)。

利用接受拒绝方法生成\(X\):
\begin{itemize} 
\item Step 1:生成\((0,1)\)上的均匀分布随机数\(U_1\),令\(Y=U_1\)
\item Step 2:生成\((0,1)\)上的均匀分布随机数\(U_2\)
\item Step 3: 如果\(U_2 \leq 16 Y^2(Y-1)^2\),令\(X=Y\),否则返回Step 1.
\end{itemize}

\subsubsection{数值试验}

得到结果如下图所示,重复操作10万次,得到长度为53298的分布如题干所示的随机数\(X\),且均值、方差分别为\(0.4998224,0.03593751\)。并且\(\frac{1}{c}=0.53333\),实际运算效率为\(0.53298\)

\centerline{\includegraphics[width=5.5in]{H3_4.eps}}


\subsection{method 2: the inverse transform method}

\subsubsection{计算方法}

令\(Y\)为均匀分布,其概率密度函数\(g(x)=-6x(x-1),  0<x<1\),这样有\(c= max (\frac{f(x)}{g(x)})=\frac{5}{4}\)。

利用接受拒绝方法生成\(X\):
\begin{itemize} 
\item Step 1:生成\((0,1)\)上的均匀分布随机数\(U_1\)
\item Step 2:求解\(2Y^3+3Y^2=U_1\),生成\(Y \);
\item Step 3:生成\((0,1)\)上的均匀分布随机数\(U_2\)
\item Step 4:如果\(U_2 \leq -4Y(Y-1)\),令\(X=Y\),否则返回Step1.

\end{itemize}

\subsubsection{数值试验}

得到结果如下图所示,重复操作1万次,得到长度为8007的分布如题干所示的随机数\(X\),且均值、方差分别为\(0.4979851,0.03592379\)。并且\(\frac{1}{c}=0.8\),实际运算效率为\(0.8007\)

\centerline{\includegraphics[width=5.5in]{H3_4_2.eps}}


\section{第五题}  

给出产生具有如下概率密度函数的随机变量的接受拒绝方法,\(f(x)=\frac{1}{2}x^2e^{-x},0 \leq x <\infty\).假设用指数分布来产生此分布,给出最优的参数\(\lambda\).



\subsection{计算最优参数}
令\(Y\)为指数分布,其概率密度函数为\(g(x)=\lambda e^{-\lambda x}\),令\(c(x,\lambda) = max(\frac{f(x)}{g(x)}=\frac{1}{2\lambda}x^2 e^{-x+\lambda x})\)。

对\(c(x,\lambda)\)关于\(x\)求导:
\[
[2x+(\lambda-1)]e^{-x+\lambda x}=0 
\]

得到
\[
x=\frac{2}{1-\lambda}
\]

带入\(c(x,\lambda)\)得到
\[c(\lambda)=\frac{1}{\lambda}\frac{2}{(1-\lambda)^2}e^{-2}
\]
对\(\lambda\)求导令其为零,得到
\[
\lambda=1,\mbox{  or  },\lambda=\frac{1}{3}
\]

最终有\(\lambda= \frac{1}{3}\)为最优参数,此时\(c=\frac{27}{2}e^{-2}\)

\subsection{计算方法}
按照the acceptance-rejection method 生成\(X\):

\begin{itemize} 
\item Step 1:生成指数分布Y,其概率密度函数为\(g(x)=\frac{1}{3} e^{-\frac{1}{3} x}\)。
\item Step 2:生成\((0,1)\)上的均匀分布随机数,\(U\)
\item Step 3: 如果\(U \leq \frac{1}{9} Y^2 e^{2-\frac{2}{3}Y}\),令\(X=Y\),否则返回Step 1.
\end{itemize}

其中利用the inverse transform algorithm生成指数分布Y步骤如下:
\begin{itemize} 
\item Step 1:生成\((0,1)\)上均匀分布随机数\(U\) 
\item Step 2:令\(Y = -3 \ln(1-U)\)。
\end{itemize}


\subsection{数值试验}

得到结果如下图所示,重复操作10万次,得到长度为18094的分布如题干所示的随机数\(X\),且均值、方差分别为\(3.007682,2.975097 \)。并且\(\frac{1}{c}=0.5473375\),实际有效操作比例为\(0.54957\)

\centerline{\includegraphics[width=5.5in]{H3_5.eps}}

\end{document}